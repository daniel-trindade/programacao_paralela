\documentclass[a4paper, 12pt]{article}
\usepackage[top=1.8cm, bottom=1.8cm, left=1.5cm, right=1.5cm]{geometry}
\usepackage{amsmath}
\usepackage{listings}
\usepackage{float}
\usepackage{tikz}


\begin{document}
	\begin{center}
		Universidade Federal do Rio Grande do Norte
		
		Departamento de Engenharia da Computação e Automação
		
		DCA3703 - Programação Paralela
		
		\textbf{Tarefa 9 - Regiões críticas nomeadas e Locks explícitos}
		
		\textbf{Aluno:} Daniel Bruno Trindade da Silva
	\end{center}
	
	\section{Introdução}
	O presente trabalho tem como objetivo explorar o uso de paralelismo com a API OpenMP para manipulação segura de estruturas de dados compartilhadas — especificamente listas encadeadas — em ambientes concorrentes. Foram desenvolvidos dois programas que criam tarefas paralelas para realizar inserções em listas encadeadas, garantindo a integridade dos dados e evitando condições de corrida.
	
	O primeiro programa lida com duas listas encadeadas, cada uma associada a uma região crítica nomeada. Esse arranjo permite que múltiplas inserções ocorram simultaneamente, desde que sejam em listas diferentes, promovendo paralelismo eficiente. Já o segundo programa generaliza o cenário anterior para um número arbitrário de listas definido pelo usuário. Nesse caso, torna-se inviável o uso de regiões críticas nomeadas, sendo necessário empregar locks explícitos para garantir a exclusão mútua sem comprometer o desempenho.
	
	A seguir, serão apresentados os dois programas, acompanhados de uma análise das estratégias de sincronização utilizadas e das razões pelas quais o uso de locks se torna essencial na versão generalizada.

	
	
\end{document}