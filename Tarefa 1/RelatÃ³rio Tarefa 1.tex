\documentclass[a4paper, 12pt]{article}
\usepackage[top=2cm, bottom=2cm, left=2.5cm, right=2.5cm]{geometry}

\begin{document}
	\begin{center}
		Universidade Federal do Rio Grande do Norte
		
		Departamento de Engenharia da Computação e Automação
		
		DCA3703 - Programação Paralela
		
		\textbf{Tarefa 1 - Memória Cache}
		
		\textbf{Aluno:} Daniel Bruno Trindade da Silva
	\end{center}
	
	\vspace{1cm}
	
	\textbf{Introdução:}
	
	Nesta tarefa, exploramos a multiplicação de matriz por vetor (MxV), com o objetivo de analisar como dois diferentes padrões de acesso aos valores da matriz (por linhas e por colunas) impactam no tempo de execução e o motivo pelo qual as execuções diferem.
	
	
	A tarefa consiste em implementar duas versões do algoritmo em C, uma que percorre a matriz por linhas (laço externo sobre linhas, interno sobre colunas) e outra que acessa os elementos por colunas (laço externo sobre colunas, interno sobre linhas). Ao executar ambas as versões faremos a medição de tempo, com o fim de comparar o desempenho das duas abordagens em matrizes de diferentes tamanhos, identificando a partir de qual dimensão os tempos de execução passam a divergir significativamente.
	
	\vspace{1cm}
	
	\textbf{Implementação:}
	
	Para testar as duas abordagens e obter seus resultados em um único código, implementamos uma função para cada uma delas: \texttt{mat\_vec\_row}, que realiza o acesso aos elementos da matriz por linhas, e \texttt{mat\_vet\_col}, que acessa os elementos por colunas. Com o código estruturado dessa forma, podemos testar ambas as abordagens em uma única execução.
	
	Faremos a chamada de ambas as funções em um laço de repetição com o fim de fazer testes aumentando o numero de elementos da matriz e do vetor em cada repetição para identificar a partir de qual dimensão começamos a ter uma grande divergência no tempo de execução.
	
	Para execução da análise, realizamos a multiplicação da matriz com tamanho \textit{n$\times$n} por um vetor também de tamanho \textit{n} com \textit{n} assumindo os valores de 100, 1000, 10000, 50000 e 100000. Em cada uma das execuções são medidos os tempos de tomados por cada uma das abordagens propostas e entregues em segundos.
	
	\vspace{1cm}
	
	\textbf{Código:}
	
	O código utiliza a biblioteca \texttt{stdlib.h} para alocação de memória e a \texttt{time.h} para aferição do tempo de execução. O código ficou como se segue:
	
	\begin{verbatim}
		#include <stdio.h>
		#include <stdlib.h>
		#include <time.h>
		
		// Função para medir o tempo
		double get_time(clock_t start, clock_t end) {
			return (double)(end - start) / CLOCKS_PER_SEC;
		}
		
		// Multiplicacao MxV com acesso por linhas
		void mat_vec_row(double** mat, double* vec, double* res, int n) {
			for (int i = 0; i < n; i++) {
				res[i] = 0.0;
				for (int j = 0; j < n; j++) {
					res[i] += mat[i][j] * vec[j];
				}
			}
		}
		
		// Multiplicacao MxV com acesso por colunas
		void mat_vec_col(double** mat, double* vec, double* res, int n) {
			for (int j = 0; j < n; j++) {
				for (int i = 0; i < n; i++) {
					res[i] += mat[i][j] * vec[j];
				}
			}
		}
		
		int main() {
			int sizes[] = {100, 500, 1000, 2000, 3000};
			int num_tests = 5;
			
			for (int t = 0; t < num_tests; t++) {
				int n = sizes[t];
				printf("Tamanho da matriz: %d\n", n);
				
				// Alocação de matriz e vetores
				double** mat = (double**) malloc(n * sizeof(double*));
				double* vec = (double*) malloc(n * sizeof(double));
				double* res_row = (double*) calloc(n, sizeof(double));
				double* res_col = (double*) calloc(n, sizeof(double));
				
				for (int i = 0; i < n; i++) {
					mat[i] = (double*) malloc(n * sizeof(double));
					for (int j = 0; j < n; j++) {
						mat[i][j] = (double)rand() / RAND_MAX;
					}
					vec[i] = (double)rand() / RAND_MAX;
				}
				
				// Tempo para acesso por linhas
				clock_t start = clock();
				mat_vec_row(mat, vec, res_row, n);
				clock_t end = clock();
				printf("Tempo (acesso por linhas): %.4f s\n", get_time(start, end));
				
				// Tempo para acesso por colunas
				start = clock();
				mat_vec_col(mat, vec, res_col, n);
				end = clock();
				printf("Tempo (acesso por colunas): %.4f s\n\n", get_time(start, end));
				
				// Liberação de memória
				for (int i = 0; i < n; i++) {
					free(mat[i]);
				}
				free(mat);
				free(vec);
				free(res_row);
				free(res_col);
			}
			
			return 0;
		}
		
	\end{verbatim}
	
	
	\textbf{Resultados:}
	
	Após execução do código obtivemos os seguintes resultados:
	
	\begin{table}[ht]
		\centering
		\begin{tabular}{|c|c|c|}
			\hline
			\textbf{valor de \textit{n}} & \textbf{Tempo de execução por linhas} & \textbf{Tempo de execução por colunas}  \\
			\hline
			100 & 0,0002s & 0,0002s \\
			500 & 0,0045s & 0,0085s \\
			1.000 & 0,0094s & 0,0215s \\
			5.000 & 0,0940s & 0,3941s \\
			10.000 & 0,3517s & 2,1709s \\
			15.000 & 0,7934s & 6,1648s \\
			\hline
		\end{tabular}
		\caption{Tempo de Execução par cada valor de \textit{n}}
	\end{table}
	
	Como é possível observar, o tempo de execução na abordagem por coluna, se torna muito superior ao da abordagem por linhas. Com \textit{n=500} o tempo da versão por colunas já quase o dobro do tempo da por linhas. No teste com \textit{n=15.000} o tempo usado pelo função que percorre as colunas é quase 8 vezes o tempo usado pela execução por linhas. Podemos observar melhor com o seguinte gráfico:
	
	
\end{document}