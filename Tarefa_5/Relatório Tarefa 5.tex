\documentclass[a4paper, 12pt]{article}
\usepackage[top=2cm, bottom=2cm, left=1.5cm, right=1.5cm]{geometry}
\usepackage{graphicx}
\usepackage{float}
\usepackage{pgfplots}


\begin{document}
	\begin{center}
		Universidade Federal do Rio Grande do Norte
		
		Departamento de Engenharia da Computação e Automação
		
		DCA3703 - Programação Paralela
		
		\textbf{Tarefa 5 - Comparação entre programação sequencial e paralela}
		
		\textbf{Aluno:} Daniel Bruno Trindade da Silva
	\end{center}
	
	\section{Introdução}
	\hspace{.7cm}Nesta prática, buscamos comparar a programação paralela com a sequencial, para isso desenvolvemos um programa capaz de contar quantos números primos existem entre 2 e um dado \textit{n} e o implementamos em uma versão sequencial e outra paralelizada de forma que não foi alterada a lógica do programa. Por fim comparamos o desempenho das versões do código para entender o impacto da paralelização no tempo de execução.
	
	\section{Metodologia}
	\hspace{.7cm}Para a versão de execução sequencial, implementamos o código com uma  função chamada \texttt{ehPrimo()} que recebe um número e retorna \texttt{true} caso seja primo, \texttt{false} caso não:
	
	\begin{verbatim}
		bool ehPrimo(int num) {
			   if (num < 2) return false;
			   for (int i = 2; i * i <= num; i++) {
				      if (num % i == 0) return false;
			   }
			   return true;
		}
	\end{verbatim}
	
	Na \texttt{main()} foi implementado um laço de repetição que passa por todos os inteiros de 2 a \textit{n} testando o valor com a função \texttt{ehPrimo()}. Nesse caso temos dois laços de repetição aninhados que testam os números um a um.
	
	\begin{verbatim}
		int main() {
		    int contador = 0;
		    struct timeval inicio, fim;
		    double time_lapsed;
		    gettimeofday(&inicio, NULL);
		    for (int i = 2; i <= N; i++) {
		        if (ehPrimo(i)) {
		            contador++;
		        }
		    }
		    
		    gettimeofday(&fim, NULL); 
		    time_lapsed = get_time(inicio, fim);
		    printf("Quantidade de números primos eh: %d\n", contador);
		    printf("Tempo gasto: %f segundos\n", time_lapsed);
		    
		    return 0;
		}
	\end{verbatim}
	
	Para a versão do código paralela, mantivemos a lógica implementada na sequencial, porém acrescentamos a diretiva \texttt{\#pragma omp parallel for} que paraleliza a execução do laço de repetição, distribuindo entre os threads disponíveis. Devido a condição de corrida estabelecida pelo paralelismo do laço precisamos adicionar uma área crítica na variável contador, e nesse caso adicionamos a diretiva \texttt{reduction(+:contador)} que protege a variável desse problema. Com as alterações necessárias a \texttt{main()} ficou da seguinte forma:
	
	\begin{verbatim}
		int main() {
		    int contador = 0;
		    struct timeval inicio, fim;
		    double time_lapsed;
		    gettimeofday(&inicio, NULL);
		    
		    #pragma omp parallel for reduction(+:contador)
		    
		    for (int i = 2; i <= N; i++) {
		        if (ehPrimo(i)) {
		            contador++;
		        }
		    }
			
		    gettimeofday(&fim, NULL); 
		    time_lapsed = get_time(inicio, fim);
		    printf("Quantidade de números primos eh: %d\n", contador);
		    printf("Tempo gasto: %f segundos\n", time_lapsed);
			
		    return 0;
		}
	\end{verbatim}
	
	
	Em ambas as versões o tempo de execução foi calculado utilizando o método \texttt{gettimeofday} da biblioteca \texttt{sys/time.h} para que possamos comparar seus resultados.
	
	\section{Resultados}
	
	Sequencial
	
	n - Tempo de Execução
	
	100 - 0,000011
	1000 - 0,000183
	10000 - 0,002914
	100000 - 0,057376
	1000000 - 0,292285
	10000000 - 5,107619
	100000000 - 140,143158
	
	Paralelo com 4 threads
	
	n - Tempo de Execução
	
	100 - 0,000011
	1000 - 0,000142
	10000 - 0,001346
	100000 - 0,050420
	1000000 - 0,271619
	10000000 - 4,971940
	100000000 - 142,141867
	
\end{document}