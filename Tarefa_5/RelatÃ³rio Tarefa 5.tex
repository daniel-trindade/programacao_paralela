\documentclass[a4paper, 12pt]{article}
\usepackage[top=2cm, bottom=2cm, left=1.5cm, right=1.5cm]{geometry}
\usepackage{graphicx}
\usepackage{float}
\usepackage{pgfplots}


\begin{document}
	\begin{center}
		Universidade Federal do Rio Grande do Norte
		
		Departamento de Engenharia da Computação e Automação
		
		DCA3703 - Programação Paralela
		
		\textbf{Tarefa 5 - Comparação entre programação sequencial e paralela}
		
		\textbf{Aluno:} Daniel Bruno Trindade da Silva
	\end{center}
	
	\section{Introdução:}
	\hspace{.7cm}Nesta prática, buscamos comparar a programação paralela com a sequencial, para isso desenvolvemos um programa capaz de contar quantos números primos existem entre 2 e um dado \textit{n} e o implementamos com uma versão sequencial e outra paralelizada de forma que não foi alterada a lógica do programa. Por fim comparamos o desempenho das versões do código para entender o impacto da paralelização no tempo de execução. 
\end{document}