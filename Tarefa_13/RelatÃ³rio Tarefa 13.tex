\documentclass[a4paper, 12pt]{article}
\usepackage[top=1.5cm, bottom=1.5cm, left=1.5cm, right=1.5cm]{geometry}
\usepackage{float}
\usepackage[utf8]{inputenc}
\usepackage{array}
\usepackage{geometry}
\usepackage{placeins}

\begin{document}
	\begin{center}
		Universidade Federal do Rio Grande do Norte
		
		Departamento de Engenharia da Computação e Automação  
		
		DCA3703 - Programação Paralela  
		
		\textbf{Tarefa 13: Afinidade de threads}  
		
		\textbf{Aluno:} Daniel Bruno Trindade da Silva  
	\end{center}  
	
	\section{Introdução}
	
	\hspace{0.62cm}Este relatório tem como objetivo apresentar os conhecimentos adquiridos durante a realização da Tarefa 13 da disciplina de \textbf{Computação Paralela}. A atividade consistiu em avaliar a escalabilidade do programa desenvolvido na Tarefa 11 — um simulador da velocidade de um fluido utilizando a equação de Navier-Stokes — aplicando diferentes políticas de afinidade de \textit{threads}.
	
	\section{Enunciado}
	
	\hspace{0.62cm}Avalie como a escalabilidade do seu código de Navier-Stokes muda ao utilizar os diversos tipos de afinidades de \textit{threads} suportados pelo sistema operacional e pelo OpenMP, no mesmo nó de computação do NPAD utilizado para a Tarefa 12.
	
	\section{Desenvolvimento}
	
	\hspace{0.62cm}Na Tarefa 11, desenvolvemos duas versões de um programa para simular a velocidade de um fluido: uma versão sequencial (serial) e outra paralelizada com OpenMP. Para a análise requerida nesta tarefa, utilizamos a versão paralelizada do código.
	
	\hspace{0.62cm}Nesta tarefa, analisamos os impactos da cláusula \texttt{proc\_bind()} com as seguintes políticas de afinidade:
	
	\begin{itemize}
		\item \textbf{\texttt{spread}} — O openMP distribui as \textit{threads} de forma espalhada pelos processadores, maximizando a distância entre elas. O objetivo é utilizar o máximo de recursos do hardware possível, como diferentes \textit{sockets} ou núcleos físicos. Aumenta a latência de comunicação entre as \textit{threads} em troca de menos disputa por recurso. Ideal quando temos programas com \textit{threads} independentes;
		
		\item \textbf{\texttt{close}} — O openMP agrupa as \textit{threads} próximas umas das outras, preferencialmente no mesmo \textit{socket} ou núcleos adjacentes ao da \textit{thread} master. Utilizada quando queremos manter as \textit{threads} próximas pra diminuir a latência, mas não queremos sobrecarregar um mesmo núcleo;
		
		\item \textbf{\texttt{master}} — Nessa política todas as \textit{threads} são alocadas no mesmo local onde a \textit{thread} principal (\textit{master/primary}) está executando. Utilizado quando a comunicação entre \textit{threads} é muito intensa e isso compença a perda de paralelismo. Pois ela gera disputa por recursos, mas reduz a latência de comunicação entre as \textit{threads};
		
		\item \textbf{\texttt{true}} — herda a política de afinidade da região paralela pai. Se não houver região pai, comporta-se de acordo com a política padrão definida pela implementação.
		
		\item \textbf{\texttt{false}} — Desativa a afinidade de \textit{threads} o escalonador do sistema operacional é quem vai decidir onde cada \textit{thread} ficará. Com isso pode haver migração de \textit{threads} entre núcleos causando mais overhead e menos previsibilidade;
	\end{itemize}
	
	\hspace{0.62cm}Reorganizamos o código de forma a possibilitar o teste de todas as políticas em uma única execução. Assim como realizado na Tarefa 12, o código foi executado com 1, 2, 4, 8, 16 e 32 \textit{threads}, para que ao final pudéssemos analisar se houve influência dessas políticas de afinidade na eficiência do código.
	
	O código foi executado no super computador da universidade utilizando o nó com o processador intel-128, cada teste foi executado 6 vezes para termos a certeza da constância dos resultados.
	
	\section{Resultados}
	Tendo em vista que a ideia é observar como as politicas de afinidade de \textit{thread} afetam a escalabilidade do nosso programa, faremos uma tabela de escalabilidade para cada politica o metodo para criação dessa tabela já foi visto anteriormente na tarefa 12.
	
	
	\subsection{Spread}
	\begin{table}[h!]
		\centering
		\begin{tabular}{|c|c|c|c|c|c|c|}
			\hline
			\textbf{\# cores} & \textbf{20x20x20} & \textbf{20x20x40} & \textbf{20x20x80} & \textbf{20x20x160} & \textbf{20x20x320} & \textbf{20x20x640} \\ \hline
			1                 & 1.00              & 1.00              & 1.00              & 1.00               & 1.00               & 1.00               \\ \hline
			2                 & 0.45              & 0.46              & 0.48              & 0.49               & 0.50               & 0.50               \\ \hline
			4                 & 0.19              & 0.22              & 0.23              & 0.24               & 0.21               & 0.25               \\ \hline
			8                 & 0.07              & 0.09              & 0.11              & 0.11               & 0.11               & 0.12               \\ \hline
			16                & 0.02              & 0.04              & 0.05              & 0.05               & 0.06               & 0.06               \\ \hline
			32                & 0.01              & 0.01              & 0.02              & 0.02               & 0.03               & 0.03               \\ \hline
		\end{tabular}
		\caption{Tabela de Escalabilidade para Política Spread}
		\label{tab:Tabela de Escalabilidade para Política Spread} % Label alterado para unicidade
	\end{table}
	
	\FloatBarrier
	
	\subsection{Close}
	\begin{table}[h!]
		\centering
		\begin{tabular}{|c|c|c|c|c|c|c|}
			\hline
			\textbf{\# cores} & \textbf{20x20x20} & \textbf{20x20x40} & \textbf{20x20x80} & \textbf{20x20x160} & \textbf{20x20x320} & \textbf{20x20x640} \\ \hline
			1                 & 1.00              & 1.00              & 1.00              & 1.00               & 1.00               & 1.00               \\ \hline
			2                 & 0.44              & 0.47              & 0.47              & 0.49               & 0.51               & 0.50               \\ \hline
			4                 & 0.19              & 0.22              & 0.23              & 0.24               & 0.25               & 0.25               \\ \hline
			8                 & 0.07              & 0.09              & 0.10              & 0.11               & 0.12               & 0.12               \\ \hline
			16                & 0.02              & 0.04              & 0.04              & 0.05               & 0.06               & 0.06               \\ \hline
			32                & 0.01              & 0.01              & 0.02              & 0.02               & 0.03               & 0.03               \\ \hline
		\end{tabular}
		\caption{Tabela de Escalabilidade para Política Close}
		\label{tab:Tabela de Escalabilidade para Política Close}
	\end{table}

	
	\FloatBarrier
	
	\subsection{Master}
	\begin{table}[h!]
		\centering
		\begin{tabular}{|c|c|c|c|c|c|c|}
			\hline
			\textbf{\# cores} & \textbf{20x20x20} & \textbf{20x20x40} & \textbf{20x20x80} & \textbf{20x20x160} & \textbf{20x20x320} & \textbf{20x20x640} \\ \hline
			1                 & 1.00              & 1.00              & 1.00              & 1.00               & 1.00               & 1.00               \\ \hline
			2                 & 0.01              & 0.01              & 0.03              & 0.06               & 0.11               & 0.12               \\ \hline
			4                 & 0.00              & 0.00              & 0.01              & 0.01               & 0.02               & 0.04               \\ \hline
			8                 & 0.07              & 0.09              & 0.11              & 0.11               & 0.11               & 0.12               \\ \hline
			16                & 0.02              & 0.04              & 0.05              & 0.05               & 0.06               & 0.06               \\ \hline
			32                & 0.01              & 0.01              & 0.02              & 0.02               & 0.03               & 0.03               \\ \hline
		\end{tabular}
		\caption{Tabela de Escalabilidade para Política Master}
		\label{tab:Tabela de Escalabilidade para Política Master} % Alterado para um novo label
	\end{table}
	
	\vspace{5cm}
	
	\FloatBarrier
	
	\subsection{False}
	\begin{table}[h!]
		\centering
		\begin{tabular}{|c|c|c|c|c|c|c|}
			\hline
			\textbf{\# cores} & \textbf{20x20x20} & \textbf{20x20x40} & \textbf{20x20x80} & \textbf{20x20x160} & \textbf{20x20x320} & \textbf{20x20x640} \\ \hline
			1                 & 1.00              & 1.00              & 1.00              & 1.00               & 1.00               & 1.00               \\ \hline
			2                 & 0.83              & 0.86              & 0.91              & 0.91               & 0.90               & 0.96               \\ \hline
			4                 & 0.45              & 0.48              & 0.49              & 0.49               & 0.50               & 0.56               \\ \hline
			8                 & 0.10              & 0.14              & 0.17              & 0.19               & 0.21               & 0.24               \\ \hline
			16                & 0.05              & 0.07              & 0.08              & 0.09               & 0.10               & 0.12               \\ \hline
			32                & 0.03              & 0.04              & 0.04              & 0.05               & 0.05               & 0.06               \\ \hline
		\end{tabular}
		\caption{Tabela de Escalabilidade para Política False}
		\label{tab:Tabela de Escalabilidade para Política False} % Label alterado para unicidade
	\end{table}
	
	
	\FloatBarrier
	
	\subsection{True}
	\begin{table}[h!]
		\centering
		\begin{tabular}{|c|c|c|c|c|c|c|}
			\hline
			\textbf{\# cores} & \textbf{20x20x20} & \textbf{20x20x40} & \textbf{20x20x80} & \textbf{20x20x160} & \textbf{20x20x320} & \textbf{20x20x640} \\ \hline
			1                 & 1.00              & 1.00              & 1.00              & 1.00               & 1.00               & 1.00               \\ \hline
			2                 & 0.44              & 0.47              & 0.48              & 0.49               & 0.49               & 0.50               \\ \hline
			4                 & 0.19              & 0.22              & 0.23              & 0.24               & 0.25               & 0.25               \\ \hline
			8                 & 0.07              & 0.09              & 0.11              & 0.11               & 0.12               & 0.12               \\ \hline
			16                & 0.02              & 0.04              & 0.05              & 0.05               & 0.06               & 0.06               \\ \hline
			32                & 0.01              & 0.01              & 0.02              & 0.02               & 0.02               & 0.03               \\ \hline
		\end{tabular}
		\caption{Insira a legenda da sua tabela aqui}
		\label{tab:minha_tabela}
	\end{table}	
	
	\FloatBarrier
		
	\section{Analise dos Resultados}
	
	As politicas de afinidade de \textit{thread} foram testadas na sequencia em que as tabelas foram montadas. Essas tabelas mostram o comportamento da escalabilidade relacionando o tamanho do problema com a quantidade de \textit{thread} que a executaram, medindo através da sua eficiência.
	
	\subsection{Spread, True e Close}
	Essas três políticas apresentaram comportamentos muito similares, caracterizados por eficiências bastante baixas em praticamente todas as configurações. A eficiência caiu rapidamente com o aumento do número de threads, atingindo valores próximos a zero com 16 e 32 threads. O aumento do tamanho do problema trouxe pouca ou nenhuma melhora significativa na eficiência, indicando baixa escalabilidade fraca.
	
	\subsection{Master}
	A política MASTER apresentou a pior eficiência dentre todas as políticas avaliadas. Com 2 e 4 threads, as eficiências foram próximas de zero. Apenas a partir de 8 threads observou-se alguma melhora, embora ainda em níveis muito baixos (em torno de 0,11 a 0,12). A ampliação do tamanho do problema não resultou em ganhos expressivos.
	
	Esse resultado era totalmente esperado, uma vez que essa política centraliza a execução na thread principal, criando um gargalo que impede a distribuição eficiente da carga de trabalho. Assim, a política MASTER é reconhecidamente inadequada para programas paralelos que visam alta escalabilidade, conforme foi comprovado experimentalmente
	
	\subsection{False}
	Esta política apresentou as melhores eficiências gerais em todas as configurações, especialmente com 2 e 4 threads, onde atingiu valores próximos a 0,9, indicando bom aproveitamento do paralelismo. Apesar de a eficiência cair progressivamente com o aumento do número de threads, ela se manteve superior às demais políticas. O aumento do tamanho do problema resultou em melhoras discretas na eficiência, evidenciando um comportamento típico de escalabilidade fraca.
	
	Era esperado que a ausência de afinidade permitisse ao sistema operacional otimizar a distribuição das threads, minimizando contenções e promovendo melhor balanceamento de carga. De fato, isso foi confirmado pelos resultados. O desempenho atendeu às expectativas.
	
	\section{Conclusão}

	A análise realizada demonstrou que as políticas de afinidade de threads exercem um impacto significativo na eficiência e escalabilidade de programas paralelos. Os experimentos confirmaram que a escolha inadequada da política pode comprometer substancialmente o desempenho da aplicação.
	
	De forma geral, os resultados obtidos foram coerentes com a literatura e com o comportamento esperado para cada política, destacando a necessidade de uma análise criteriosa do perfil computacional de cada aplicação antes de definir a política de afinidade de threads. A escalabilidade forte se mostrou limitada em todas as políticas quando o número de threads foi elevado excessivamente, enquanto a escalabilidade fraca apresentou melhoras modestas, especialmente na política FALSE.
	
	Por fim, esta avaliação reforça a importância de realizar experimentos empíricos para orientar decisões de configuração em sistemas paralelos, visando sempre o melhor aproveitamento dos recursos computacionais disponíveis.
	
	

	 

\end{document}
