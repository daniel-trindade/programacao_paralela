\documentclass[a4paper, 12pt]{article}
\usepackage[top=1.8cm, bottom=1.8cm, left=1.5cm, right=1.5cm]{geometry}
\usepackage{float}



\begin{document}
	\begin{center}
		Universidade Federal do Rio Grande do Norte
		
		Departamento de Engenharia da Computação e Automação  
		
		DCA3703 - Programação Paralela  
		
		\textbf{Tarefa 13: Afinidade de threads}  
		
		\textbf{Aluno:} Daniel Bruno Trindade da Silva  
	\end{center}  
	
	\section{Introdução}  
	\hspace{.62cm}Este relatório tem como objetivo apresentar o conhecimento adquirido durante a realização da Tarefa 13 da disciplina de \textbf{Computação Paralela}. A atividade consistiu em avaliar a escalabilidade do programa desenvolvido na tarefa 11 (Simulador de velocidade de um fluido utilizando a equação de Navier-Stokes) aplicando diferentes politicas de afinidade de \textit{threads}.
	
	\section{Enunciado}    
	\hspace{.62cm} Avalie como a escalabilidade do seu código de Navier-Strokes muda ao utilizar os diversos tipos de afinidades de threads suportados pelo sistema operacional e pelo OpenMP no mesmo nó de computação do NPAD que utilizou para a tarefa 12.
	
	\section{Desenvolvimento}
	\hspace{.62cm}Na Tarefa 11, desenvolvemos duas versões de um programa para simular a velocidade de um fluido: uma versão sequencial (serial) e outra paralelizada com OpenMP. Para a análise requerida nessa tarefa utilizaremos a versão paralelizada do código.
	
	A analise alvo dessa tarefa requer que utilizemos a cláusula \texttt{proc\_bind()}, com os seguintes parametros:
	
	\begin{itemize}
		\item \textbf{\texttt{spread}}
		\item \textbf{\texttt{close}}
		\item \textbf{\texttt{master}}
		\item \textbf{\texttt{true}}
		\item \textbf{\texttt{false}}
	\end{itemize}  
	
	 

\end{document}
