\documentclass[a4paper, 12pt]{article}
\usepackage[top=1.8cm, bottom=1.8cm, left=1.5cm, right=1.5cm]{geometry}
\usepackage{float}
\usepackage[utf8]{inputenc}
\usepackage[brazil]{babel}
\usepackage{array}
\usepackage{geometry}
\usepackage{placeins}
\geometry{a4paper, margin=2cm}



\begin{document}
	\begin{center}
		Universidade Federal do Rio Grande do Norte
		
		Departamento de Engenharia da Computação e Automação  
		
		DCA3703 - Programação Paralela  
		
		\textbf{Tarefa 13: Afinidade de threads}  
		
		\textbf{Aluno:} Daniel Bruno Trindade da Silva  
	\end{center}  
	
	\section{Introdução}
	
	\hspace{0.62cm}Este relatório tem como objetivo apresentar os conhecimentos adquiridos durante a realização da Tarefa 13 da disciplina de \textbf{Computação Paralela}. A atividade consistiu em avaliar a escalabilidade do programa desenvolvido na Tarefa 11 — um simulador da velocidade de um fluido utilizando a equação de Navier-Stokes — aplicando diferentes políticas de afinidade de \textit{threads}.
	
	\section{Enunciado}
	
	\hspace{0.62cm}Avalie como a escalabilidade do seu código de Navier-Stokes muda ao utilizar os diversos tipos de afinidades de \textit{threads} suportados pelo sistema operacional e pelo OpenMP, no mesmo nó de computação do NPAD utilizado para a Tarefa 12.
	
	\section{Desenvolvimento}
	
	\hspace{0.62cm}Na Tarefa 11, desenvolvemos duas versões de um programa para simular a velocidade de um fluido: uma versão sequencial (serial) e outra paralelizada com OpenMP. Para a análise requerida nesta tarefa, utilizamos a versão paralelizada do código.
	
	\hspace{0.62cm}Nesta tarefa, analisamos os impactos da cláusula \texttt{proc\_bind()} com as seguintes políticas de afinidade:
	
	\begin{itemize}
		\item \textbf{\texttt{spread}} — O openMP distribui as \textit{threads} de forma espalhada pelos processadores, maximizando a distância entre elas. O objetivo é utilizar o máximo de recursos do hardware possível, como diferentes \textit{sockets} ou núcleos físicos. Aumenta a latência de comunicação entre as \textit{threads} em troca de menos disputa por recurso. Ideal quando temos programas com \textit{threads} independentes;
		
		\item \textbf{\texttt{close}} — O openMP agrupa as \textit{threads} próximas umas das outras, preferencialmente no mesmo \textit{socket} ou núcleos adjacentes ao da \textit{thread} master. Utilizada quando queremos manter as \textit{threads} próximas pra diminuir a latência, mas não queremos sobrecarregar um mesmo núcleo;
		
		\item \textbf{\texttt{master}} — Nessa política todas as \textit{threads} são alocadas no mesmo local onde a \textit{thread} principal (\textit{master/primary}) está executando. Utilizado quando a comunicação entre \textit{threads} é muito intensa e isso compença a perda de paralelismo. Pois ela gera disputa por recursos, mas reduz a latência de comunicação entre as \textit{threads};
		
		\item \textbf{\texttt{true}} — herda a política de afinidade da região paralela pai. Se não houver região pai, comporta-se de acordo com a política padrão definida pela implementação.
		
		\item \textbf{\texttt{false}} — Desativa a afinidade de \textit{threads} o escalonador do sistema operacional é quem vai decidir onde cada \textit{thread} ficará. Com isso pode haver migração de \textit{threads} entre núcleos causando mais overhead e menos previsibilidade;
	\end{itemize}
	
	\hspace{0.62cm}Reorganizamos o código de forma a possibilitar o teste de todas as políticas em uma única execução. Assim como realizado na Tarefa 12, o código foi executado com 1, 2, 4, 8, 16 e 32 \textit{threads}, para que ao final pudéssemos analisar se houve influência dessas políticas de afinidade na eficiência do código.
	
	O código foi executado no super computador da universidade utilizando o nó com o processador intel-128, cada teste foi executado 6 vezes para termos a certeza da constância dos resultados.
	
	\section{Resultados}
	Tendo em vista que a ideia é observar como as politicas de afinidade de \textit{thread} afetam a escalabilidade do nosso programa, faremos uma tabela de escalabilidade para cada politica o metodo para criação dessa tabela já foi visto anteriormente na tarefa 12.
	
	\subsection{Execução sem diretiva \texttt{PROC\_BIND()}}	
	\begin{table}[htbp]
		\centering
		\begin{tabular}{|c|c|c|c|c|c|c|}
			\hline
			\textbf{\# cores }& \textbf{20x20x20} & \textbf{20x20x40} & \textbf{20x20x80} & \textbf{20x20x160} & \textbf{20x20x320} & \textbf{20x20x640} \\ \hline
			1        & 1.00     & 1.00     & 1.00     & 1.00      & 1.00      & 1.00      \\ \hline
			2        & 0.84     & 0.86     & 0.90     & 0.91      & 0.92      & 0.96      \\ \hline
			4        & 0.67     & 0.71     & 0.76     & 0.78      & 0.81      & 0.83      \\ \hline
			8        & 0.45     & 0.51     & 0.56     & 0.59      & 0.66      & 0.65      \\ \hline
			16       & 0.25     & 0.32     & 0.39     & 0.44      & 0.35      & 0.50      \\ \hline
			32       & 0.11     & 0.15     & 0.19     & 0.22      & 0.25      & 0.27      \\ \hline
		\end{tabular}
		\caption{Tabela de escalabilidade da política default.}
	\end{table}
	
	\FloatBarrier
	
	\subsection{Spread}
	\begin{table}[htbp]
		\centering
		\begin{tabular}{|c|c|c|c|c|c|c|}
			\hline
			\textbf{\# cores} & \textbf{20x20x20} & \textbf{20x20x40} & \textbf{20x20x80} & \textbf{20x20x160} & \textbf{20x20x320} & \textbf{20x20x640} \\
			\hline
			\textbf{1} & 1,00 & 1,00 & 1,00 & 1,00 & 1,00 & 1,00 \\
			\hline
			\textbf{2} & 0,85 & 0,87 & 0,90 & 0,91 & 0,93 & 0,96 \\
			\hline
			\textbf{4} & 0,67 & 0,72 & 0,76 & 0,78 & 0,80 & 0,83 \\
			\hline
			\textbf{8} & 0,46 & 0,52 & 0,57 & 0,59 & 0,66 & 0,65 \\
			\hline
			\textbf{16} & 0,26 & 0,33 & 0,39 & 0,44 & 0,49 & 0,50 \\
			\hline
			\textbf{32} & 0,11 & 0,16 & 0,19 & 0,22 & 0,25 & 0,27 \\
			\hline
		\end{tabular}
		\caption{Tabela de escalabilidade da política Spread}
	\end{table}
	
	\vspace{5cm}
	
	\FloatBarrier
	
	\subsection{Close}
	\begin{table}[htbp]
		\centering
		\begin{tabular}{|c|c|c|c|c|c|c|}
			\hline
			\textbf{\# cores} & \textbf{20x20x20} & \textbf{20x20x40} & \textbf{20x20x80} & \textbf{20x20x160} & \textbf{20x20x320} & \textbf{20x20x640} \\
			\hline
			\textbf{1} & 1,00 & 1,00 & 1,00 & 1,00 & 1,00 & 1,00 \\
			\hline
			\textbf{2} & 0,85 & 0,87 & 0,90 & 0,91 & 0,93 & 0,96 \\
			\hline
			\textbf{4} & 0,67 & 0,72 & 0,76 & 0,78 & 0,81 & 0,83 \\
			\hline
			\textbf{8} & 0,45 & 0,52 & 0,57 & 0,59 & 0,66 & 0,65 \\
			\hline
			\textbf{16} & 0,25 & 0,33 & 0,39 & 0,44 & 0,35 & 0,50 \\
			\hline
			\textbf{32} & 0,11 & 0,16 & 0,19 & 0,22 & 0,26 & 0,27 \\
			\hline
		\end{tabular}
		\caption{Tabela de escalabilidade da política Close}
	\end{table}
	
	\FloatBarrier
	
	\subsection{Master}
	\begin{table}[htbp]
		\centering
		\begin{tabular}{|c|c|c|c|c|c|c|}
			\hline
			\textbf{\# cores} & \textbf{20x20x20} & \textbf{20x20x40} & \textbf{20x20x80} & \textbf{20x20x160} & \textbf{20x20x320} & \textbf{20x20x640} \\
			\hline
			\textbf{1} & 1,00 & 1,00 & 1,00 & 1,00 & 1,00 & 1,00 \\
			\hline
			\textbf{2} & 0,85 & 0,87 & 0,90 & 0,91 & 0,93 & 0,96 \\
			\hline
			\textbf{4} & 0,67 & 0,72 & 0,76 & 0,78 & 0,80 & 0,83 \\
			\hline
			\textbf{8} & 0,45 & 0,52 & 0,57 & 0,59 & 0,65 & 0,65 \\
			\hline
			\textbf{16} & 0,25 & 0,33 & 0,39 & 0,44 & 0,35 & 0,50 \\
			\hline
			\textbf{32} & 0,11 & 0,15 & 0,19 & 0,22 & 0,26 & 0,27 \\
			\hline
		\end{tabular}
		\caption{Tabela de escalabilidade da política Master}
	\end{table}
	
	\FloatBarrier
	
	\section{Analise dos Resultados}
	

	
	
	

	 

\end{document}
