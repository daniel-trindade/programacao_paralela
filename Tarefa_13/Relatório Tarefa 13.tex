\documentclass[a4paper, 12pt]{article}
\usepackage[top=1.8cm, bottom=1.8cm, left=1.5cm, right=1.5cm]{geometry}
\usepackage{float}



\begin{document}
	\begin{center}
		Universidade Federal do Rio Grande do Norte
		
		Departamento de Engenharia da Computação e Automação  
		
		DCA3703 - Programação Paralela  
		
		\textbf{Tarefa 13: Afinidade de threads}  
		
		\textbf{Aluno:} Daniel Bruno Trindade da Silva  
	\end{center}  
	
	\section{Introdução}
	
	\hspace{0.62cm}Este relatório tem como objetivo apresentar os conhecimentos adquiridos durante a realização da Tarefa 13 da disciplina de \textbf{Computação Paralela}. A atividade consistiu em avaliar a escalabilidade do programa desenvolvido na Tarefa 11 — um simulador da velocidade de um fluido utilizando a equação de Navier-Stokes — aplicando diferentes políticas de afinidade de \textit{threads}.
	
	\section{Enunciado}
	
	\hspace{0.62cm}Avalie como a escalabilidade do seu código de Navier-Stokes muda ao utilizar os diversos tipos de afinidades de \textit{threads} suportados pelo sistema operacional e pelo OpenMP, no mesmo nó de computação do NPAD utilizado para a Tarefa 12.
	
	\section{Desenvolvimento}
	
	\hspace{0.62cm}Na Tarefa 11, desenvolvemos duas versões de um programa para simular a velocidade de um fluido: uma versão sequencial (serial) e outra paralelizada com OpenMP. Para a análise requerida nesta tarefa, utilizamos a versão paralelizada do código.
	
	\hspace{0.62cm}Nesta tarefa, analisamos os impactos da cláusula \texttt{proc\_bind()} com as seguintes políticas de afinidade:
	
	\begin{itemize}
		\item \textbf{\texttt{spread}} — distribui as \textit{threads} de forma espalhada pelos processadores, maximizando a distância entre elas. O objetivo é utilizar o máximo de recursos de hardware possível, como diferentes \textit{sockets}, nós NUMA ou núcleos físicos.
		
		\item \textbf{\texttt{close}} — agrupa as \textit{threads} próximas umas das outras, preferencialmente no mesmo \textit{socket}, nó NUMA ou núcleos adjacentes.
		
		\item \textbf{\texttt{master}} — todas as \textit{threads} são alocadas no mesmo local onde a \textit{thread} principal (\textit{master/primary}) está executando.
		
		\item \textbf{\texttt{true}} — herda a política de afinidade da região paralela pai. Se não houver região pai, comporta-se de acordo com a política padrão definida pela implementação.
		
		\item \textbf{\texttt{false}} — não estabelece nenhuma política de afinidade específica; as \textit{threads} podem migrar livremente entre processadores durante a execução.
	\end{itemize}
	
	\hspace{0.62cm}Reorganizamos o código de forma a possibilitar o teste de todas as políticas em uma única execução. Assim como realizado na Tarefa 12, o código foi executado com 1, 2, 4, 8, 16 e 32 \textit{threads}, para que ao final pudéssemos analisar se houve influência dessas políticas de afinidade na eficiência do código.
	
	O código foi executado no super computador da universidade utilizando o nó com o processador intel-128, cada teste foi executado 6 vezes para termos a certeza da constância dos resultados.
	
	\section{Resultados}

	 

\end{document}
