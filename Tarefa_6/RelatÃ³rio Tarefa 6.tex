\documentclass[a4paper, 12pt]{article}
\usepackage[top=2cm, bottom=2cm, left=1.5cm, right=1.5cm]{geometry}
\usepackage{graphicx}
\usepackage{float}
\usepackage{pgfplots}


\begin{document}
	\begin{center}
		Universidade Federal do Rio Grande do Norte
		
		Departamento de Engenharia da Computação e Automação
		
		DCA3703 - Programação Paralela
		
		\textbf{Tarefa 6 - Escopo de variáveis e regiões críticas}
		
		\textbf{Aluno:} Daniel Bruno Trindade da Silva
	\end{center}
	
	\section{Introdução}
	\hspace{.7cm}Esta tarefa tem como objetivo entender paralelismo com OpenMP, o impacto das cláusulas de escopo e como o não uso dessas clausulas pode causar conflito nas variáveis devido a condição de corrida. Para isso implementaremos um algoritmo estocástico para estimativa do número $\pi$ usando o método de Monte Carlo.
	
	Inicialmente, a tarefa propõe a paralelização do algoritmo utilizando a diretiva \texttt{\#pragma omp parallel} for. No entanto, essa abordagem incorreta pode resultar em comportamentos inesperados devido ao compartilhamento inadequado de variáveis entre as threads, ocasionando condições de corrida (race conditions). O relatório analisa os motivos que levam a esse resultado incorreto e propõe uma reestruturação do código utilizando as diretivas \texttt{\#pragma omp parallel} em conjunto com \texttt{\#pragma omp for}, além da aplicação das cláusulas private, firstprivate, lastprivate, shared e default(none).
	
	
\end{document}